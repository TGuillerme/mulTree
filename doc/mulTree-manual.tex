\documentclass{article}\usepackage[]{graphicx}\usepackage[]{color}
%% maxwidth is the original width if it is less than linewidth
%% otherwise use linewidth (to make sure the graphics do not exceed the margin)
\makeatletter
\def\maxwidth{ %
  \ifdim\Gin@nat@width>\linewidth
    \linewidth
  \else
    \Gin@nat@width
  \fi
}
\makeatother

\definecolor{fgcolor}{rgb}{0.345, 0.345, 0.345}
\newcommand{\hlnum}[1]{\textcolor[rgb]{0.686,0.059,0.569}{#1}}%
\newcommand{\hlstr}[1]{\textcolor[rgb]{0.192,0.494,0.8}{#1}}%
\newcommand{\hlcom}[1]{\textcolor[rgb]{0.678,0.584,0.686}{\textit{#1}}}%
\newcommand{\hlopt}[1]{\textcolor[rgb]{0,0,0}{#1}}%
\newcommand{\hlstd}[1]{\textcolor[rgb]{0.345,0.345,0.345}{#1}}%
\newcommand{\hlkwa}[1]{\textcolor[rgb]{0.161,0.373,0.58}{\textbf{#1}}}%
\newcommand{\hlkwb}[1]{\textcolor[rgb]{0.69,0.353,0.396}{#1}}%
\newcommand{\hlkwc}[1]{\textcolor[rgb]{0.333,0.667,0.333}{#1}}%
\newcommand{\hlkwd}[1]{\textcolor[rgb]{0.737,0.353,0.396}{\textbf{#1}}}%

\usepackage{framed}
\makeatletter
\newenvironment{kframe}{%
 \def\at@end@of@kframe{}%
 \ifinner\ifhmode%
  \def\at@end@of@kframe{\end{minipage}}%
  \begin{minipage}{\columnwidth}%
 \fi\fi%
 \def\FrameCommand##1{\hskip\@totalleftmargin \hskip-\fboxsep
 \colorbox{shadecolor}{##1}\hskip-\fboxsep
     % There is no \\@totalrightmargin, so:
     \hskip-\linewidth \hskip-\@totalleftmargin \hskip\columnwidth}%
 \MakeFramed {\advance\hsize-\width
   \@totalleftmargin\z@ \linewidth\hsize
   \@setminipage}}%
 {\par\unskip\endMakeFramed%
 \at@end@of@kframe}
\makeatother

\definecolor{shadecolor}{rgb}{.97, .97, .97}
\definecolor{messagecolor}{rgb}{0, 0, 0}
\definecolor{warningcolor}{rgb}{1, 0, 1}
\definecolor{errorcolor}{rgb}{1, 0, 0}
\newenvironment{knitrout}{}{} % an empty environment to be redefined in TeX

\usepackage{alltt}
\usepackage[sc]{mathpazo}
\usepackage[T1]{fontenc}
\usepackage{geometry}
\usepackage{natbib}
\usepackage[utf8]{inputenc}
\geometry{verbose,tmargin=2.5cm,bmargin=2.5cm,lmargin=2.5cm,rmargin=2.5cm}
\setcounter{secnumdepth}{2}
\setcounter{tocdepth}{2}
\usepackage{url}
\usepackage[unicode=true,pdfusetitle,
 bookmarks=true,bookmarksnumbered=true,bookmarksopen=true,bookmarksopenlevel=2,
 breaklinks=false,pdfborder={0 0 1},backref=false,colorlinks=false]
 {hyperref}
\hypersetup{
 pdfstartview={XYZ null null 1}}

\newcommand{\dispRity}{\texttt{dispRity} }
\newcommand{\R}{\texttt{R} }
\IfFileExists{upquote.sty}{\usepackage{upquote}}{}
\begin{document}



\title{\texttt{mulTree} manual}


\author{Thomas Guillerme}

\maketitle

This package is based on the \href{http://cran.r-project.org/web/packages/MCMCglmm/index.html}{\texttt{MCMCglmm} package} and runs a MCMCglmm analysis on multiple trees.

\section{Installation}
You can install the latest released version (1.2) directly from GitHub using the following:

\begin{knitrout}
\definecolor{shadecolor}{rgb}{0.969, 0.969, 0.969}\color{fgcolor}\begin{kframe}
\begin{alltt}
\hlkwa{if}\hlstd{(}\hlopt{!}\hlkwd{require}\hlstd{(devtools))} \hlkwd{install.packages}\hlstd{(}\hlstr{"devtools"}\hlstd{)}
\hlkwd{install_github}\hlstd{(}\hlstr{"TGuillerme/mulTree"}\hlstd{,} \hlkwc{ref} \hlstd{=} \hlstr{"release"}\hlstd{)}
\end{alltt}
\end{kframe}
\end{knitrout}

%\tableofcontents

\section{Quick go through}

Note that this section will be developed in the future.
Stay tuned!

\begin{knitrout}
\definecolor{shadecolor}{rgb}{0.969, 0.969, 0.969}\color{fgcolor}\begin{kframe}
\begin{alltt}
\hlcom{## Loading the package}
\hlkwd{library}\hlstd{(mulTree)}
\end{alltt}


{\ttfamily\noindent\itshape\color{messagecolor}{\#\# Loading required package: caper\\\#\# Loading required package: MASS\\\#\# Loading required package: mvtnorm\\\#\# Loading required package: coda\\\#\# Loading required package: hdrcde\\\#\# hdrcde 3.1 loaded\\\#\# \\\#\# Loading required package: MCMCglmm\\\#\# Loading required package: Matrix}}\begin{alltt}
\hlcom{## Loading the lifespan data}
\hlkwd{data}\hlstd{(lifespan)}
\end{alltt}
\end{kframe}
\end{knitrout}


\subsection{Combining phylogenies: \texttt{tree.bind}}
This function allows to combine phylogenies, in this example, we are going to combine the \texttt{trees\_mammalia} and the \texttt{trees\_aves} randomly at a root.age of 250 million years ago.

\begin{knitrout}
\definecolor{shadecolor}{rgb}{0.969, 0.969, 0.969}\color{fgcolor}\begin{kframe}
\begin{alltt}
\hlcom{## The 2 mammalian trees}
\hlstd{trees_mammalia}
\end{alltt}
\begin{verbatim}
## 2 phylogenetic trees
\end{verbatim}
\begin{alltt}
\hlcom{## Number of tips in both}
\hlkwd{unlist}\hlstd{(}\hlkwd{lapply}\hlstd{(trees_mammalia, Ntip))}
\end{alltt}
\begin{verbatim}
## [1] 134 134
\end{verbatim}
\begin{alltt}
\hlcom{## The 2 aves trees}
\hlstd{trees_aves}
\end{alltt}
\begin{verbatim}
## 2 phylogenetic trees
\end{verbatim}
\begin{alltt}
\hlcom{## Number of tips in both}
\hlkwd{unlist}\hlstd{(}\hlkwd{lapply}\hlstd{(trees_aves, Ntip))}
\end{alltt}
\begin{verbatim}
## [1] 58 58
\end{verbatim}
\begin{alltt}
\hlcom{## Combining them}
\hlstd{combined_trees} \hlkwb{<-} \hlkwd{tree.bind}\hlstd{(trees_mammalia, trees_aves,} \hlkwc{sample} \hlstd{=} \hlnum{2}\hlstd{,}
    \hlkwc{root.age} \hlstd{=} \hlnum{250}\hlstd{)}
\hlstd{combined_trees}
\end{alltt}
\begin{verbatim}
## 2 phylogenetic trees
\end{verbatim}
\begin{alltt}
\hlcom{## Number of tips in both}
\hlkwd{unlist}\hlstd{(}\hlkwd{lapply}\hlstd{(combined_trees, Ntip))}
\end{alltt}
\begin{verbatim}
## [1] 192 192
\end{verbatim}
\end{kframe}
\end{knitrout}

\subsection{Preparing the \texttt{mulTree} data: \texttt{as.mulTree}}
This function allows to combine the trees to some trait data and get it in the right format to be passed to the \texttt{mulTree} function.

\begin{knitrout}
\definecolor{shadecolor}{rgb}{0.969, 0.969, 0.969}\color{fgcolor}\begin{kframe}
\begin{alltt}
\hlcom{## The trait data}
\hlkwd{head}\hlstd{(lifespan_volant)}
\end{alltt}
\begin{verbatim}
##                species    class  longevity       mass    volant
## 1 Dolichotis_patagonum Mammalia -0.1490041  1.0875446 nonvolant
## 2       Eidolon_helvum Mammalia  0.4686111 -0.2748337    volant
## 3      Elephas_maximus Mammalia  2.1071286  3.1220340 nonvolant
## 4         Equus_asinus Mammalia  1.6128024  2.0352764 nonvolant
## 5     Equus_burchellii Mammalia  1.2962194  2.2295299 nonvolant
## 6       Equus_caballus Mammalia  1.9001076  2.2548716 nonvolant
\end{verbatim}
\begin{alltt}
\hlcom{## Creating the mulTree object}
\hlstd{mulTree_data} \hlkwb{<-} \hlkwd{as.mulTree}\hlstd{(}\hlkwc{data} \hlstd{= lifespan_volant,} \hlkwc{tree} \hlstd{= combined_trees,}
    \hlkwc{taxa} \hlstd{=} \hlstr{"species"}\hlstd{)}

\hlcom{## This object is classified as "mulTree" and contains different elements to be}
\hlcom{## passed to the mulTree function}
\hlkwd{class}\hlstd{(mulTree_data) ;} \hlkwd{names}\hlstd{(mulTree_data)}
\end{alltt}
\begin{verbatim}
## [1] "mulTree"
## [1] "phy"          "data"         "random.terms" "taxa.column"
\end{verbatim}
\end{kframe}
\end{knitrout}


\subsection{Running the \texttt{MCMCglmm} on multiple trees: \texttt{mulTree}}
This function intakes the normal arguments form the \href{http://cran.r-project.org/web/packages/MCMCglmm/index.html}{\texttt{MCMCglmm} function} alongside with the \texttt{mulTree} object.

\textbf{WARNING: this part of the code can take several minutes to run!}
\begin{knitrout}
\definecolor{shadecolor}{rgb}{0.969, 0.969, 0.969}\color{fgcolor}\begin{kframe}
\begin{alltt}
\hlcom{## The glmm formula}
\hlstd{my_formula} \hlkwb{<-} \hlstd{longevity} \hlopt{~} \hlstd{mass} \hlopt{+} \hlstd{volant}

\hlcom{## The MCMC parameters (generations, sampling, burnin)}
\hlstd{my_parameters} \hlkwb{<-} \hlkwd{c}\hlstd{(}\hlnum{100000}\hlstd{,} \hlnum{10}\hlstd{,} \hlnum{1000}\hlstd{)}

\hlcom{## The MCMCglmm priors}
\hlstd{my_priors} \hlkwb{<-} \hlkwd{list}\hlstd{(}\hlkwc{R} \hlstd{=} \hlkwd{list}\hlstd{(}\hlkwc{V} \hlstd{=} \hlnum{1}\hlopt{/}\hlnum{2}\hlstd{,} \hlkwc{nu} \hlstd{=} \hlnum{0.002}\hlstd{),}
    \hlkwc{G} \hlstd{=} \hlkwd{list}\hlstd{(}\hlkwc{G1} \hlstd{=} \hlkwd{list}\hlstd{(}\hlkwc{V} \hlstd{=} \hlnum{1}\hlopt{/}\hlnum{2}\hlstd{,} \hlkwc{nu} \hlstd{=} \hlnum{0.002}\hlstd{)))}

\hlcom{## Running the MCMCglmm on multiple trees}
\hlkwd{mulTree}\hlstd{(}\hlkwc{mulTree.data} \hlstd{= mulTree_data,} \hlkwc{formula} \hlstd{= my_formula,}
    \hlkwc{parameters} \hlstd{= my_parameters,} \hlkwc{priors} \hlstd{= my_priors,}
    \hlkwc{output} \hlstd{=} \hlstr{"longevity_example"}\hlstd{,} \hlkwc{ESS} \hlstd{=} \hlnum{50}\hlstd{,} \hlkwc{chains} \hlstd{=} \hlnum{2}\hlstd{)}
\end{alltt}


{\ttfamily\noindent\bfseries\color{errorcolor}{\#\# Error in as.vector(data): no method for coercing this S4 class to a vector}}\end{kframe}
\end{knitrout}

\subsection{Reading the models: \texttt{read.mulTree}}
The models where written out of the \texttt{R} environment in your current directory.
To reinput them in the \texttt{R} environment and analysis the results, we can use the \texttt{read.mulTree} function.

\begin{knitrout}
\definecolor{shadecolor}{rgb}{0.969, 0.969, 0.969}\color{fgcolor}\begin{kframe}
\begin{alltt}
\hlcom{## Reading only one specific model}
\hlstd{one_model} \hlkwb{<-} \hlkwd{read.mulTree}\hlstd{(}\hlstr{"longevity_example-tree1_chain1"}\hlstd{,} \hlkwc{model} \hlstd{=} \hlnum{TRUE}\hlstd{)}
\end{alltt}


{\ttfamily\noindent\bfseries\color{errorcolor}{\#\# Error: File "{}longevity\_example-tree1\_chain1"{} not found in current directory.}}\begin{alltt}
\hlcom{## This model is a normal MCMCglmm object that has been ran on one single tree}
\hlkwd{class}\hlstd{(one_model) ;} \hlkwd{names}\hlstd{(one_model)}
\end{alltt}


{\ttfamily\noindent\bfseries\color{errorcolor}{\#\# Error in eval(expr, envir, enclos): object 'one\_model' not found}}

{\ttfamily\noindent\bfseries\color{errorcolor}{\#\# Error in eval(expr, envir, enclos): object 'one\_model' not found}}\begin{alltt}
\hlcom{## Reading the convergence diagnosis test to see if the two chains converged for}
\hlcom{## each tree}
\hlkwd{read.mulTree}\hlstd{(}\hlstr{"longevity_example"}\hlstd{,} \hlkwc{convergence} \hlstd{=} \hlnum{TRUE}\hlstd{)}
\end{alltt}


{\ttfamily\noindent\bfseries\color{errorcolor}{\#\# Error: File "{}longevity\_example"{} not found in current directory.}}\begin{alltt}
\hlcom{## As indicated here, the chains converged for both chains!}

\hlcom{## Reading all the models to perform the MCMCglmm analysis on multiple trees}
\hlstd{all_models} \hlkwb{<-} \hlkwd{read.mulTree}\hlstd{(}\hlstr{"longevity_example"}\hlstd{)}
\end{alltt}


{\ttfamily\noindent\bfseries\color{errorcolor}{\#\# Error: File "{}longevity\_example"{} not found in current directory.}}\begin{alltt}
\hlkwd{str}\hlstd{(all_models)}
\end{alltt}


{\ttfamily\noindent\bfseries\color{errorcolor}{\#\# Error in str(all\_models): object 'all\_models' not found}}\begin{alltt}
\hlcom{## This object contains 39600 estimations of the Intercept and the terms!}

\hlcom{## Removing the chains from the current directory}
\hlkwd{file.remove}\hlstd{(}\hlkwd{list.files}\hlstd{(}\hlkwc{pattern}\hlstd{=}\hlstr{"longevity_example"}\hlstd{))}
\end{alltt}
\begin{verbatim}
## logical(0)
\end{verbatim}
\end{kframe}
\end{knitrout}

\subsection{Summarising the results: \texttt{summary.mulTree}}
It is possible to summarise the results of the glmm on all chains using the \texttt{summary.mulTree} function.

\begin{knitrout}
\definecolor{shadecolor}{rgb}{0.969, 0.969, 0.969}\color{fgcolor}\begin{kframe}
\begin{alltt}
\hlcom{## Summarising the results by estimating the highest density regions}
\hlcom{## and their associated 95 and 50 confidence intervals (default)}
\hlstd{summarised_results} \hlkwb{<-} \hlkwd{summary}\hlstd{(all_models)}
\end{alltt}


{\ttfamily\noindent\bfseries\color{errorcolor}{\#\# Error in summary(all\_models): error in evaluating the argument 'object' in selecting a method for function 'summary': Error: object 'all\_models' not found}}\begin{alltt}
\hlstd{summarised_results}
\end{alltt}


{\ttfamily\noindent\bfseries\color{errorcolor}{\#\# Error in eval(expr, envir, enclos): object 'summarised\_results' not found}}\begin{alltt}
\hlcom{## Summarising the results using the quick 'n' dirty way along with some options}
\hlcom{## i.e just measuring the distributions quantiles}
\hlcom{## note that there is a S3 method for "mulTree" objects allowing to just use}
\hlcom{## summary()}
\hlkwd{summary}\hlstd{(all_models,} \hlkwc{use.hdr} \hlstd{=} \hlnum{FALSE}\hlstd{,} \hlkwc{cent.tend} \hlstd{= mean,} \hlkwc{prob} \hlstd{=} \hlkwd{c}\hlstd{(}\hlnum{75}\hlstd{,} \hlnum{25}\hlstd{))}
\end{alltt}


{\ttfamily\noindent\bfseries\color{errorcolor}{\#\# Error in summary(all\_models, use.hdr = FALSE, cent.tend = mean, prob = c(75, : error in evaluating the argument 'object' in selecting a method for function 'summary': Error: object 'all\_models' not found}}\end{kframe}
\end{knitrout}

\subsection{Visualising the results: \texttt{plot.mulTree}}
Finally it is possible to simply plot the results from the MCMCglmm analysis on multiple trees using the S3 method for \texttt{plot.mulTree}.

\begin{knitrout}
\definecolor{shadecolor}{rgb}{0.969, 0.969, 0.969}\color{fgcolor}\begin{kframe}
\begin{alltt}
\hlcom{## Plotting using the default options}
\hlkwd{plot}\hlstd{(summarised_results)}
\end{alltt}


{\ttfamily\noindent\bfseries\color{errorcolor}{\#\# Error in plot(summarised\_results): object 'summarised\_results' not found}}\end{kframe}
\end{knitrout}

\begin{knitrout}
\definecolor{shadecolor}{rgb}{0.969, 0.969, 0.969}\color{fgcolor}\begin{kframe}
\begin{alltt}
\hlcom{## Plotting using some more pretty options}
\hlkwd{plot}\hlstd{(summarised_results,} \hlkwc{horizontal} \hlstd{=} \hlnum{TRUE}\hlstd{,} \hlkwc{ylab} \hlstd{=} \hlstr{""}\hlstd{,} \hlkwc{cex.coeff} \hlstd{=} \hlnum{0.8}\hlstd{,}
    \hlkwc{main} \hlstd{=} \hlstr{"Posterior distributions"}\hlstd{,} \hlkwc{ylim} \hlstd{=} \hlkwd{c}\hlstd{(}\hlopt{-}\hlnum{2}\hlstd{,}\hlnum{2}\hlstd{),} \hlkwc{cex.terms} \hlstd{=} \hlnum{0.5}\hlstd{,}
    \hlkwc{terms} \hlstd{=} \hlkwd{c}\hlstd{(}\hlstr{"Intercept"}\hlstd{,} \hlstr{"Body Mass"}\hlstd{,} \hlstr{"Volancy"}\hlstd{,} \hlstr{"Phylogeny"}\hlstd{,} \hlstr{"Residuals"}\hlstd{),}
    \hlkwc{col} \hlstd{=} \hlstr{"red"}\hlstd{,} \hlkwc{cex.main} \hlstd{=} \hlnum{0.8}\hlstd{)}
\end{alltt}


{\ttfamily\noindent\bfseries\color{errorcolor}{\#\# Error in plot(summarised\_results, horizontal = TRUE, ylab = "{}"{}, cex.coeff = 0.8, : object 'summarised\_results' not found}}\begin{alltt}
\hlkwd{abline}\hlstd{(}\hlkwc{v} \hlstd{=} \hlnum{0}\hlstd{,} \hlkwc{lty} \hlstd{=} \hlnum{3}\hlstd{)}
\end{alltt}


{\ttfamily\noindent\bfseries\color{errorcolor}{\#\# Error in int\_abline(a = a, b = b, h = h, v = v, untf = untf, ...): plot.new has not been called yet}}\end{kframe}
\end{knitrout}

\end{document}
